%%%%%%%%%%%%%%%%%%%%%%%%%%%%%%%%%%%%%%%%%
% Lachaise Assignment
% LaTeX Template
% Version 1.0 (26/6/2018)
%
% This template originates from:
% http://www.LaTeXTemplates.com
%
% Authors:
% Marion Lachaise & François Févotte
% Vel (vel@LaTeXTemplates.com)
%
% License:
% CC BY-NC-SA 3.0 (http://creativecommons.org/licenses/by-nc-sa/3.0/)
% 
%%%%%%%%%%%%%%%%%%%%%%%%%%%%%%%%%%%%%%%%%

%----------------------------------------------------------------------------------------
%	PACKAGES AND OTHER DOCUMENT CONFIGURATIONS
%----------------------------------------------------------------------------------------

\documentclass{article}

\input{structure.tex} % Include the file specifying the document structure and custom commands

%----------------------------------------------------------------------------------------
%	ASSIGNMENT INFORMATION
%----------------------------------------------------------------------------------------

\title{Synthèse d'image : compte rendu} % Title of the assignment

\author{Valentin VERSTRACTE \& Evan PETIT}

\date{L3 --- \today} % University, school and/or department name(s) and a date

%----------------------------------------------------------------------------------------

\begin{document}

\maketitle % Print the title

%----------------------------------------------------------------------------------------
%	INTRODUCTION
%----------------------------------------------------------------------------------------

\section*{Introduction} % Unnumbered section


%----------------------------------------------------------------------------------------
%	Répertoire
%----------------------------------------------------------------------------------------

\section{Une conception orienté objet} % Numbered section

Chaque primitive, courbe de bézier et courbe paramétrique sont représenter par des classes. Chaqu'une d'elle dérivée d'une classe objet qui va nous permettre de connaître et d'effectuer la translation

\begin{figure}[htp]
    \centering
    \includegraphics[width=5cm]{./assets/class_hierarchy.png}
    \caption{Diagramme de classes}
    \label{fig:classe}
\end{figure}


%------------------------------------------------

\section{Les ailes, deux courbes de bézier}
\begin{figure}[htp]
    \centering
    \includegraphics[width=10cm]{./assets/kig_bezier.png}
    \caption{Courbe de bézier sur kig}
    \label{fig:classe}
\end{figure}

\begin{figure}[htp]
    \centering
    \includegraphics[width=5cm]{./assets/openGL_bezier.png}
    \caption{Courbe de bézier sur openGL}
    \label{fig:classe}
\end{figure}


%------------------------------------------------

\section{Le cylindre paramétrique}

%------------------------------------------------

\section{Les textures}

%------------------------------------------------

\section{Les animations}

\subsection{Une animation automatique}

L'animation automatique se porte les ailes du dragon. On incrémente un angle de + ou - 25 ° se qui donne l'impression de voler. La logique algorithmique est plûtot simple. On incrémente un tout petit l'angle dans la fonction anim ( appeler par glut une fois qu'il ne fait rien ). Si l'angle est supérieur à 25 on décremente. Si l'angle est inférieur à -25 on incrémente. 
 
\subsection{Une animation manuelle}

L'animation n'est pas très impressionnante. On peut juste baisser ou lever la queue ... On fonction des touches h et n.

%------------------------------------------------

\section{Les touches disponibles}

\begin{itemize}
\item \keystroke{p}: affichage du carré plein
\item \keystroke{f} : affichage du mode de fil de fer
\item \keystroke{s} : affichage en mode de sommets seuls
\item \keystroke{z} : permet de zoomer
\item \keystroke{Z} : permet de dézoomer 
\item \keystroke{h} : élève la queue du dragon
\item \keystroke{n} : abaisse la queue du dragon 
\item \keystroke{q} : quitter l'application 
\item \UArrow \DArrow \LArrow \RArrow : déplace la caméra en haut, en bas, à droite, à gauche

\end{itemize}

%----------------------------------------------------------------------------------------

\end{document}
